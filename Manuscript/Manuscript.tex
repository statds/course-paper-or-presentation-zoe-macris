\documentclass[12pt, letterpaper]{article}
%\documentclass[12pt, letterpaper, titlepage]{article}

\usepackage{amsmath}
\usepackage{booktabs}
\usepackage{amsthm}
\usepackage{graphicx}
\usepackage[margin=1in]{geometry}
\usepackage{hyperref}
\hypersetup{colorlinks = true, linkcolor = blue, citecolor=blue, urlcolor = blue}
\usepackage{enumitem}
\usepackage{setspace}
\usepackage{lipsum}
\usepackage{siunitx}
\usepackage[square,numbers]{natbib}
\bibliographystyle{abbrvnat}



\usepackage[]{lineno}
\linenumbers*[1]
% %% patches to make lineno work better with amsmath
\newcommand*\patchAmsMathEnvironmentForLineno[1]{%
 \expandafter\let\csname old#1\expandafter\endcsname\csname #1\endcsname
 \expandafter\let\csname oldend#1\expandafter\endcsname\csname end#1\endcsname
 \renewenvironment{#1}%
 {\linenomath\csname old#1\endcsname}%
 {\csname oldend#1\endcsname\endlinenomath}}%
\newcommand*\patchBothAmsMathEnvironmentsForLineno[1]{%
 \patchAmsMathEnvironmentForLineno{#1}%
 \patchAmsMathEnvironmentForLineno{#1*}}%

\AtBeginDocument{%
 \patchBothAmsMathEnvironmentsForLineno{equation}%
 \patchBothAmsMathEnvironmentsForLineno{align}%
 \patchBothAmsMathEnvironmentsForLineno{flalign}%
 \patchBothAmsMathEnvironmentsForLineno{alignat}%
 \patchBothAmsMathEnvironmentsForLineno{gather}%
 \patchBothAmsMathEnvironmentsForLineno{multline}%
}

% control floats
\renewcommand\floatpagefraction{.9}
\renewcommand\topfraction{.9}
\renewcommand\bottomfraction{.9}
\renewcommand\textfraction{.1}
\setcounter{totalnumber}{50}
\setcounter{topnumber}{50}
\setcounter{bottomnumber}{50}

\newcommand{\jy}[1]{\textcolor{blue}{JY: #1}}
\newcommand{\eds}[1]{\textcolor{red}{EDS: (#1)}}

% NOTE: To produce blinded version, replace "0" with "1" below.
\newcommand{\blind}{0}



\setlength{\linewidth}{3in}

\begin{document}
%\maketitle

\if0\blind
{
  \title{\bf Manuscript Draft}
  \author{Zoe Macris}
\date{December 2023}
  \maketitle} 

%\doublespace

\section{Abstract}
\label{sec:abstract}



\section{Keywords}
\label{sec:keywords}


\section{Introduction}
\label{sec:intro}

For being around for just over 200 years, the bicycle has constantly evolved in it's utility to us as a society. Originally just for hobbyists, and inhibitively expensive, they are now one of the most accessible modes of transport \cite{BIKE2023}. It is an excellent for of exercises for our increasingly sedentary society, giving a cardio workout that improves flexibility and isn't hard on the joints. It also can help people's mental health as well as physical. Beyond these individual benefits, cycling has a benefit for the environment and society as a whole It is a sustainable mode of transportation, not releasing any carbon emissions and requiring much less infrastructure and parking than automobiles. It is because of these benefits, that investing time and resources into cycling infrastructure is important. Bike lanes, separated paths and bike-friendly intersections can all be used to encourage cycling, leading to the benefits of improved public health and air quality, as well as making it a safe activity for those who chose to cycle. Additionally, cycling is only a viable option for some individuals with disabilities or limited mobility when there is proper cycling infrastructure. This is where statistics comes in. In order to reap these benefits of cycling, we can analyse data to promote cycling infrastructure to decision makers. 

\section{Data}
\label{sec:data}



\section{Methods}
\label{sec:methods}


\section{Application}
\label{sec:application}


\section{Discussion and Conclusion}
\label{sec:discandconclus}

The paper by Mateo-Babiano takes the approach of using by Spearman correlation to analyse the effects of various cycling measures on usage, as can be seen in ~\ref{fig:Mateo-Babiano}

\begin{table}[tbp]
\label{fig:Mateo-Babiano}
\centering
\caption{Length of bikeway by bikeway type in Brisbane and CityCycle areas. Spearman correlation of station usage and infrastructure type: from Mateo-Babiano paper}
\begin{tabular}{l*{5}{p{2cm}}}
\toprule
{Infrastructure type} & {Length Brisbane Area (km)} & {Lenth in CityCycle area (km)} & {Correlation coefficent} & {P-value}\\
\midrule
Shared Pathway & 327.9 & 18.0 & 0.42 & <0.01 \\
Bicycle Lane & 186.7 & 21.3 & 0.03 & 0.69 \\
Bicycle Awareness Zone (BAZ) & 303.7 & 40.3 & -0.07 & 0.39 \\
Bicycle Path & 23.9 & 2.6 & 0.27 & <0.01 \\
Bicycle Route (BAZ) & 76.9 & 8.2 & 0.13 & 0.11 \\
Connect & 19.5 & 5.0 & 0.16 & 0.049 \\
Informal Off Road & 65.8 & 4.6 & 0.00 & 0.98 \\
Informal On Road & 18.2 & 4.1 & -0.13 & 0.12 \\
Separated Pathway & 1.8 & 1.3 & 0.29 & <0.0 \\
\bottomrule
\end{tabular}
\end{table}


\section{Appendix}
\label{sec:appendix}


\section{Acknowledgements}
\label{sec:acknow}


\section{Appendix}
\label{sec:appendix}

\bibliography{Citations}

\end{document}
%%% LocalWords: nonparametric semiparametric autocorrelation ARMA
%%% Local Variables:
%%% mode: latex
%%% TeX-master: t
%%% ispell-personal-dictionary: ".aspell.en.pws"
%%% fill-column: 80
%%% eval: (auto-fill-mode 1)
%%% End: